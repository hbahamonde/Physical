% what's beatiful is good, politics
Dion1972a introduced among political psychologists the idea that ``What is beautiful is good''

``A person's physical appearance [...] is the personal characteristic that is most obvious and accessible to others in social interaction'' Dion1972a, 285

``physically attractive persons assumed to possess more socially desirable personalities than those of lesser attractiveness, but it is presumed that their lives will be happier and more successful.'' Dion1972a, 289

``Voters vote beautiful'' Efrain1974a, 352

``Politics represents an area of public life in which the effects of appearance may lead to very important social consequences'' Efrain1974a, 352

``Kennedy-Nixon'' Efrain1974a, 353

``Candidate evaluations are one of the most important but least understood facets of American voting behavior.'' Miller1986a, 521 *say that they note this, because the finding is from Stokes (1966).

% heuristics

``people employ a limited number of heuristics which reduce these judgments to simpler ones'' Tversky1973a, 163

``A person is said to employ the availability heuristic whenever he estimates frequency or probability by the ease with which instances or associations could be brought to mind'' Tversky1973a, 164

``This article shows that people rely on a limited number of heuristic principles which reduce the complex tasks of assessing probabilities and predicting values to simpler judgmental operations'' Tversky1974a, 3

representativeness heuristic: ``people assess the probability that [someone] is engaged in a particular occupation [assessing] the degree to which [that person] is representative of, or similar to, the stereotype of a [certain profession]'' Tversky1974a, 3, 4

``Prototypes are categories people hold about the nature of the world. An ideal president prototype in particular consists of the features that citizens believe best define an exemplary president.'' Kinder1980a, 316

``schema[:] people categorize objects and simplify information as a result of general limits to human cognitive capacity'' Miller1986a, 523

``In the absence of other information, candidates' physical attractiveness (conveyed through photographs) had a substantial influence on subjects' global evaluations of them and inferences of both their personal qualities and their political ideology'' Riggle1992a, 67

``regardless of what other attribute information is available, the stereotype will be used as a "cognitive shortcut" to the judgment'' Riggle1992a, 70

% it's not issue-related but looks-emotionality

``perceptions of candidates are generally focused on "personality" characteristics rather than on issue concerns or partisan group connections. Contrary to the implications of past research, higher education is found to be correlated with a greater likelihood of using personality categories rather than with making issue statements'' Miller1986a, 521

``it is widely believed that a candidate's personal image affects his or her chances of being elected'' Sigelman1987a, 32

``the democratic process may not be undermined by emotionality as is generally presupposed. Instead, we believe that people use emotions as tools for efficient information processing and thus enhance their abilities to engage in meaningful political deliberation'' Marcus1993a, 672


% femininity, paradox

``powerful indirect effects mediated by her perceived femininity, dynamism, niceness and age'' Sigelman1987a, 32

``This link between physical attractiveness and perceptions of sex-role related qualities poses no problem for the attractive male office-seeker, but it means that the attractive female candidate may be considered too "feminine" to possess such male-stereotyped strengths as assertiveness and boldness, which are requisites for those who serve in high-level positions.''  Sigelman1987a, 33

``To the extent that likability sways votes, a more attractive female candidate should thus be more successful than her less attractive female counterparts. Because attractiveness maximizes a woman's perceived femininity, however, it may also cause people to conclude that she lacks forcefulness and competence, thereby undermining her appeal to the voters''  Sigelman1987a, 34

% status

``physical attractiveness produces stereotyped status inferences'' Kalick1988a, 469 or ``physical attractiveness is a status cue'' (same page)

``Better-looking people sort into occupations where beauty may be more productive'' Hamermesh1994a, 1174


# HERE Hamermesh1994a

