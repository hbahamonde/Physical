% what's beatiful is good, politics

%Dion1972a introduced among political psychologists the idea that ``What is beautiful is good'' (or the ‘‘beautiful is good’’ axiom)

``physically attractive persons assumed to possess more socially desirable personalities than those of lesser attractiveness, but it is presumed that their lives will be happier and more successful.'' Dion1972a, 289

``candidate attractiveness [...] and perceived competence [...] increased a subject's likelihood of attributing her political views to a candidate'' Herrmann2016a, 401 (THIS IS CALLED `HALO EFFECT')

% ``Voters vote beautiful'' (Efrain1974a, 352) and ``[b]eauty matters'' (Rosar2008, 64).

``Politics represents an area of public life in which the effects of appearance may lead to very important social consequences'' Efrain1974a, 352


%``‘frog pond’ effect [:] if an attractive candidate competes with some ugly rivals he or she will receive a higher poll compared to a situation of competition with other attractive candidates.'' Rosar2008, 73


	% Nixon stuff

		Stockemer2019a, 748

		``Kennedy-Nixon'' Efrain1974a, 353

		``for example, Nixon’s “five-o’clock” shadow and refusal to use makeup in the 1960 debate with Kennedy is widely thought to have affected voter evaluations.'' Mattes2010a, 42


	% well-stablished relationship
		%``It is by now well established that politicians with an appealing appearance win more votes in elections'' Berggren2017, 79

		%``For quite some time, studies have found a link between attractiveness and election results'' Ditonto2018, 430

		%``dozens of studies have confirmed that the more attractive a candidate is, the more votes he or she tends to receive'' Praino2019a, 531

		%``it is widely believed that a candidate's personal image affects his or her chances of being elected'' Sigelman1987a, 32

		``Candidate evaluations are one of the most important but least understood facets of American voting behavior.'' Miller1986a, 521 *say that they note this, because the finding is from Stokes (1966).

	% it's not issue-related but looks-emotionality

		%``the democratic process may not be undermined by emotionality as is generally presupposed. Instead, we believe that people use emotions as tools for efficient information processing and thus enhance their abilities to engage in meaningful political deliberation'' Marcus1993a, 672

		%``candidates chosen as more likely to physically threaten the subjects actually lost 65\% of the real elections'' Mattes2010a, 41

		%``Research in psychology suggests that, in many cases, facial inferences about personality traits extend across cultural boundaries.'' (Lawson2010a, 565)

		%``Here, we extend these findings by showing that individuals living worlds apart, Americans and Indians, can predict elections in the new democracies of Mexico and Brazil.'' Lawson2010a, 566 *the universality of appearance judgments*

		``Looks should matter more when elections are candidate centered, not party centered, and when electoral institutions increase the costs of acquiring information about candidates, such as when there are numerous contenders for a given office.'' Lawson2010a, 566

		``physically attractive politicians seem to outperform their peers, as do good-looking people in other professions.'' (Lawson2010a, 563)

		``voters are biased by facial appearances'' (Antonakis2009a, 1183)

		%``the political ignorance of the American voter is one of the best-documented features of contemporary politics'' Bartels1996, 194

% heuristics (definition)

		%``by "heuristics," we mean problem-solving strategies'' Lau2001a, 952
	
		%``A person is said to employ the availability heuristic whenever he estimates frequency or probability by the ease with which instances or associations could be brought to mind'' Tversky1973a, 164

		%``schema[:] people categorize objects and simplify information as a result of general limits to human cognitive capacity'' Miller1986a, 523



	% heuristics: when there is no info

		%``This article shows that people rely on a limited number of heuristic principles which reduce the complex tasks of assessing probabilities and predicting values to simpler judgmental operations'' Tversky1974a, 3

		%``candidate appearance [...] is possibly the most important [heuristic]'' Lau2001a, 954

		% ``A person's physical appearance [...] is the personal characteristic that is most obvious and accessible to others in social interaction'' Dion1972a, 285

		% ``Faces are a major source of information about other people'' Todorov2005, 1623

		% ``Laboratory studies, in which subjects cast hypothetical ballots after seeing pictures of politicians' faces, suggest that voters employ this same heuristic when evaluating candidates'' (Lawson2010a, 563)

		% ``Voters usually do not know much about the biography and political agenda of the local candidates for parliament in their constituency'' Rosar2008, 64

		%``photographs as cues in low-information elections'' Banducci2008, 903

		% ``voters can compensate for a lack of information by using cognitive shortcuts'' Banducci2008, 903

		%``In the absence of other information, candidates' physical attractiveness (conveyed through photographs) had a substantial influence on subjects' global evaluations of them and inferences of both their personal qualities and their political ideology'' Riggle1992a, 67



	% heuristics when there is some (new or a lot of) information (updating).

		
		% ``power and warmth are conveyed by varying degrees of facial maturity and, in the absence of other information about a person, these static, proximate facial cues apparently loom large.'' Keating1999, 594-595

		%``available information often trumps more relevant but harder to get information. A facial image is one, and sometimes the only, available source of information about a candidate.'' Herrmann2016a, 414

		%``high frequency of elections tends to increase voter fatigue and makes it less likely that many voters will spend a considerable amount of time and effort to gather information about candidates'' Praino2014c, 1097

		``people employ a limited number of heuristics which reduce these judgments to simpler ones'' Tversky1973a, 163

		%``perceptions of candidates are generally focused on "personality" characteristics rather than on issue concerns or partisan group connections. Contrary to the implications of past research, higher education is found to be correlated with a greater likelihood of using personality categories rather than with making issue statements'' Miller1986a, 521

		``challenges the often untested assumption that cognitive "heuristics" improve the decision-making abilities of everyday voters.'' Lau2001a, 951

		%``heuristic use generally increases the probability of a correct vote by political experts but decreases the probability of a correct vote by novices.'' Lau2001a, 951

		% ``applying a variety of information "shortcuts" to make reasonable decisions with minimal cognitive effort in all aspects of their lives'' Lau2001a, 952


	
	
	% Stereotypes

		``People who know absolutely nothing about politics nonetheless know a great deal about other people and make social judgments of all types using these social stereotypes with great cognitive efficiency'' Lau2001a, 954 *here they are citing someone else's work, but still will cite Lau and Redlawsk*

		``most people have schemas or stereotypes for political leaders'' Lau2001a, 954 *here they are citing someone else's work, but still will cite Lau and Redlawsk*

		representativeness heuristic: ``people assess the probability that [someone] is engaged in a particular occupation [assessing] the degree to which [that person] is representative of, or similar to, the stereotype of a [certain profession]'' Tversky1974a, 3, 4

		``regardless of what other attribute information is available, the stereotype will be used as a "cognitive shortcut" to the judgment'' Riggle1992a, 70

		``physical attractiveness produces stereotyped status inferences'' Kalick1988a, 469

		``attractiveness and perceived competence of candidates matter for candidates’ electoral successes'' Praino2014c, 1096


	% Attributes

		``competence and honesty are the character attributes most frequently ascribed to political candidates by voters'' Mattes2010a, 44 *not their findings, but will cite them anyways*.	

		% ``attractive candidates are more likely to be attributed the qualities associated with successful politicians'' Banducci2008, 903

		``In general, physically attractive people are thought to possess more desirable personality traits that translate into other advantages. For example, good-looking people earn more over their lifetimes'' Banducci2008, 906

		``Beauty is more strongly correlated with success than either perceived competence or trustworthiness.'' (Berggren2010, 8)

		``Prototypes are categories people hold about the nature of the world. An ideal president prototype in particular consists of the features that citizens believe best define an exemplary president.'' Kinder1980a, 316


		% status

% ``self-perceived attractiveness shaped people's social class perceptions, which in turn, influenced how people responded to inequality and social hierarchies'' Belmi2014c, 145

% ``muscularity and height (in males) and physical attractiveness (in both sexes) would theoretically have correlated positively with one’s social status, and thus with one's ability to benefit from social inequality'' Price2011a, 636

``individuals who are more characterized by these traits would be less egalitarian'' Price2011a, 636

%``physical attractiveness is a status cue'' Kalick1988a, 469

%``Better-looking people sort into occupations where beauty may be more productive'' Hamermesh1994a, 1174

``occupation and education serve as signals of competence we can test if beauty has an effect of its own that is independent of its signalling competence.'' (Berggren2010, 9)

	% first impressions

		%``Not only do photographs allow voters to form first impressions of candidates, but they also provide demographic cues, which may lead to a potential bias that favors certain attributes.'' Banducci2008, 904

		%``Unfortunately, voters are anchored in first impressions and do not appropriately correct initial inferences; additional information on the candidates does not change choices by much'' (Antonakis2009a, 1183)

		%children proved more likely than adults to prefer winning candidates (Antonakis2009a)

	% time exposure/seconds
	
		%``inferences of competence based solely on facial appearance predicted the outcomes of U.S. congressional elections [1-second exposure to the faces of the candidates]'' Todorov2005, 1623

		%``judgments made at 100 milliseconds correlated highly with judgments with no time constraints'' Mattes2010a, 43













% femininity, paradox

``powerful indirect effects mediated by her perceived femininity, dynamism, niceness and age'' Sigelman1987a, 32

``This link between physical attractiveness and perceptions of sex-role related qualities poses no problem for the attractive male office-seeker, but it means that the attractive female candidate may be considered too "feminine" to possess such male-stereotyped strengths as assertiveness and boldness, which are requisites for those who serve in high-level positions.''  Sigelman1987a, 33

``To the extent that likability sways votes, a more attractive female candidate should thus be more successful than her less attractive female counterparts. Because attractiveness maximizes a woman's perceived femininity, however, it may also cause people to conclude that she lacks forcefulness and competence, thereby undermining her appeal to the voters''  Sigelman1987a, 34

``most of these experiments do not systematically consider the gender of the candidates involved'' Ditonto2018, 431



	% premium

		``people with above-average looks receive a pay premium'', Hamermesh1994a, 1181

		%``candidate attractiveness mitigates the negative electoral effects of involvement in scandal'' Stockemer2019a, 747

``document an economically and statistically significant positive correlation between the facial attractiveness of male high school graduates and their subsequent labor market earnings.'' Scholz2015, 14

``If good-looking people are more persuasive, are treated better in social interaction and achieve higher occupational success — as evidenced in a meta-study by Langlois et al. (2000) — they might also do better in politics'' (Berggren2010, 8).


%``The physical attractiveness effect is reduced if the voter undercorrects, eliminated if the voter adequately corrects, and reversed if the voter inadvertently overcorrects. Correction occurs when voters possess both the ability and motivation to correct.'' Hart2011a, 199.

``appealing-looking politicians benefit disproportionately from television exposure, primarily among less knowledgeable individuals.'' Lenz2011a, 574

%``People who are poorly informed about politics but watch a good deal of TV cast their ballots for governor and senator disproportionately on the way candidates look.'' Lenz2011a, 575

% ``The appearance effect holds, they have found, when [...] differences in image quality and other aspects of the pictures, such as visible light, are taken into account'' Lenz2011a, 575

% ``Being physically attractive matters for electoral success'' Schubert2011a, 34

% Finland

		``study the role of beauty in politics using candidate photos'' in Finland Berggren2010

		``We show that in municipal elections, a beauty increase of one standard deviation attracts about 20\% more votes for the average non-incumbent candidate on the right and about 8\% more votes for the average non-incumbent candidate on the left.'' Berggren2017, 80

		``An increase in our measure of beauty by one standard deviation is associated with an increase of 20\% in the number of votes for the average non-incumbent parliamentary candidate'' Berggren2010, 8

		``Finland, which is suitable for our anal- ysis because of its proportional electoral system with multi-member districts, personal votes and within-party competition. Such a system allows us to study whether beauty matters more for candidates on the left or for candidates on the right, since electoral “beauty premia” can be calculated separately for different parties'' Berggren2017

% Opposite/different results

		``although competence matters most for elections involving only men, attractiveness predicts winners in women-only elections.'' Ditonto2018, 430

		``Surprisingly, attractiveness was correlated with losing elections, with the effect being driven by faces of candidates who looked politically incompetent yet personally attractive.'' Mattes2010a, 42

		``But in our study, the faces chosen as most attractive were actually far more likely to lose the election—56\% of the chosen faces lost (p-value 0.000).'' Mattes2010a, 51

		``women seeking business loans were even less likely to receive funding if they were attractive'' Kuwabara2017b, 1371

		``we find that physical attractiveness plays no role in determining candidates’ vote shares'' Wigginton2021a, 1 (they say it depends ``it seems to be conditional on certain parameters such as a personalized political culture.'')